\documentclass[12pt]{article}
\usepackage[noindent]{rajeev}
\setlength{\headheight}{14.49998pt}
\usepackage{graphicx}
\usepackage{listings}
\usepackage{amsmath}
\usepackage{amssymb}
\usepackage{geometry}
\geometry{a4paper, margin=1in}
\usepackage{color}


\begin{document}
\title{Intro to AI Assignment 3 - Probabilistic Reasoning}
\author{Rajeev Atla - 208003072\\ 
Jasmin Badyal - 208003131\\
Dhvani Patel - 21006030}
\maketitle

\section{Problem 1}

\section{Problem 2}

\section{Problem 3}

\section{Problem 4}

We can model the system as a hidden Markov model.
We can model $X_t$ as a Markov chain with the states $\set{A, B, C, D, E, F}$ and transition matrix:

$$
\begin{pmatrix}
0.2 & 0.8 & 0 & 0 & 0 & 0 \\
0 & 0.2 & 0.8 & 0 & 0 & 0 \\
0 & 0 & 0.2 & 0.8 & 0 & 0 \\
0 & 0 & 0 & 0.2 & 0.8 & 0 \\
0 & 0 & 0 & 0 & 0.2 & 0.8 \\
0 & 0 & 0 & 0 & 0 & 1 \\
\end{pmatrix}
$$

In addition,
we have the observation matrices for hot and cold:

\subsection{Part 1}

We know that the rover starts at state A with probability 1,
so $P(X_1 = A) = 1$.
The initial state vector is therefore $[1, 0, 0, 0, 0, 0]^T$.




\section{Problem 5}

\end{document}
